\chapter{Zaklju�ak}

Osnovni cilj zadatka ovog diplomskog rada, koji podrazumijeva implementaciju univerzalne, pro�irive palete komandi i njeno vrednovanje u sklopu eksperimentalnog istra�ivanja, je uspje�no postignut. Prototip palete koji je nastao u sklopu ovog rada tako�er ima i svoju ina�icu za operacijske sustave Windows, �to predstavlja veliku prednost u odnosu na ista ili sli�na rje�enja koja su navedena na po�etku ovog rada. 

Jednostavan koncept na kojem po�iva implementirano rje�enje pokazao se vrlo uspje�nim, �emu svjedo�e i izvrsni rezultati istra�ivanja. Ispitanici su iskazali zadovoljstvo upotrebljivo��u ove aplikacije te �elju za njenom uporabom u svom svakodnevnom radu. Potvr�ena je signifikantna razlika i osjetno br�i rad kada je paleta na raspolaganju. Osim svoje osnovne funkcionalnosti, paleta tako�er poma�e korisnicima u u�enju tipkovni�kih pre�aca koje prethodno nisu znali napamet. Na taj na�in, rad s ra�unalom postaje jo� u�inkovitiji jer su pre�aci br�i od pozivanja palete i tra�enja odgovaraju�e komande.

Zaklju�no, jednostavnost i u�inkovitost upotrebe, te iznimno pozitivni dojmovi ispitanika, optimisti�an su pokazatelj da bi cjelovita, produkcijska i odr�avana verzija palete komandi mogla nai�i na �iroku primjenu i popularnost kod korisnika stolnih ra�unala.