\chapter{Budu�i rad na univerzalnoj paleti komandi}

Mnogi ispitanici su izrazili �elju za kori�tenjem univerzalne palete komandi u svom svakodnevnom radu. Prototip koji je kori�ten u ispitivanju ne sadr�ava mnoge funkcionalnosti koje su nu�ne kako bi se paleta mogla �iroko distribuirati kao aplikacija oko koje bi se stvorila zajednica korisnika. Neke od ideja za budu�nost ovog softvera su:

\begin{itemize}
\item Uporaba razvojnog okvira za \textit{desktop} aplikacije koji ima ve�u zajednicu i podr�ku (primjerice, \textbf{Qt} \cite{17_2019} ili \textbf{Electron} \cite{18_2019}) te mogu�nost razvoja za vi�e \textit{desktop} platformi (Windows, MacOS, Linux).
\item Razvoj ure�iva�a JSON datoteka koje definiraju komande za svaki program.
\item Razvoj centralnog \textit{web} repozitorija za JSON datoteke komandi.
\item Rje�avanje izazova i uvo�enje pobolj�anja navedenih u poglavlju \ref{sec:challenges}:
	\begin{itemize}
	\item Podr�ka funkcionalnostima koje nisu pokrivene standardnim tipkovni�kim pre�acima
	\item Ispravna detekcija bilo kojeg programa
	\end{itemize}
\item Podr�ka za automatsko pokretanje na po�etku rada sustava.
\item Podr�ka za sakrivanje aplikacije unutar \textit{system tray} dijela radne povr�ine na Windows operacijskim sustavima te na ekvivalentima u Linux i MacOS sustavima.
\item Podr�ka za nadogradnjom aplikacije.
\item Implementacija raznih postavki poput:
	\begin{itemize}
	\item Podr�ke za razli�ite jezike
	\item Veli�ine i vrste fonta u aplikaciji
	\item Maksimalnog broja komandi vidljivih u bilo kojem trenutku
	\item Pre�aca za poziv palete komandi
	\end{itemize}
\item Promid�ba aplikacije na Internetu i stvaranje zajednice koja bi pru�ila pomo� pri izradi JSON datoteka komandi i razvoju same aplikacije. U tom slu�aju, univerzalna paleta komandi bi mogla biti aplikacija otvorenog koda.
\end{itemize}