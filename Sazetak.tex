\vspace{5pt}

%:::::::::::::::::::::::::::::::::::::::::::::::::::::
%:::::::::::: HRVATSKI :::::::::::::::::::::::::::::::
U sklopu ovog rada implementirana je univerzalna i pro�iriva paleta komandi. Paleta mo�e raditi s bilo kojom aplikacijom koja podr�ava tipkovni�ke pre�ace. Poziva se na prethodno pode�eni tipkovni�ki pre�ac, prepoznaje aplikaciju koja se trenutno nalazi u fokusu te nudi pretra�ivi popis komandi koje se mogu izvr�iti unutar te aplikacije. Odabirom komande iz palete emulira se odgovaraju�i tipkovni�ki pre�ac, �ime se posljedi�no izvr�ava ciljana akcija. Paleta je pro�iriva s obzirom na jednostavno uvo�enje novih popisa komandi za razne aplikacije. Provedeno je eksperimentalno vrednovanje u�inkovitosti univerzalne palete komandi u kojem je sudjelovalo trideset ispitanika. Rezultati su pokazali zna�ajno smanjenje vremena izvr�avanja zadataka uz dostupnu paletu komandi, u odnosu na standardni pristup pretra�ivanja funkcija kroz izborni�ku strukturu. Eksperimentalni zadaci simulirali su uobi�ajeni rad s ra�unalom, zahtijevaju�i od ispitanika naizmjeni�no kori�tenje internetskog preglednika, ure�iva�a teksta te ure�iva�a slikovnih sadr�aja. Implementirano rje�enje je i kvalitativno vrednovano kori�tenjem standardiziranog upitnika SUS, �iji su rezultati pokazali pozitivne dojmove ispitnih korisnika i visoku razinu upotrebljivosti.
%:::::::::::::::::::::::::::::::::::::::::::::::::::::

\vspace{5pt}
%
\noindent \textbf{\textit{Klju�ne rije�i} --- interakcija �ovjeka i ra�unala, paleta komandi, pretraga komandi, tipkovni�ki pre�aci, svjesnost o aplikaciji} 

%:::::::::::: KRAJ HRVATSKOG DIJELA :::::::::::::::::::


%::::::::::::::::::::::::::::::::::::::::::::::::::::::
%:::::::::::: ENGLESKI ::::::::::::::::::::::::::::::::

%\vspace{-10pt}
\section*{Abstract}
\vspace{-10pt}
Within this thesis, a universal and extendable command palette has been implemented. The palette can be utilized with any application that supports keyboard shortcuts. It is invoked by making use of a previously configured keyboard shortcut, it recognizes the application which is currently in focus, and offers a searchable list of commands that can be executed within the matching application. Selecting a command from the palette emulates the corresponding keyboard shortcut, and consequently triggers a target action. The palette is extendable due to the possibility for simple introduction of new command lists for various applications. An experimental evaluation of the efficiency of the universal command palette was carried out, involving thirty participants. The obtained results revealed a significant decrease in task execution time with the command palette being available, compared to the standard menu-based command search. Experiment tasks simulated everyday computer work, requiring participants to interchangeably use the internet browser, text editor, and image editor. The implemented solution was furthermore qualitatively evaluated using the standard SUS questionnaire, whose results indicated positive impressions of the test users and high level of usability.
%:::::::::::::::::::::::::::::::::::::::::::::::::::::::

\vspace{5pt}
%
\noindent \textbf{\textit{Keywords} --- Human-Computer Interaction, command palette, command search, keyboard shortcuts, application-awareness}

%::::::::::::::::::::::::::::::::::::::::::::::::::::::
%:::::::::::: KRAJ ENGLESKOG DIJELA :::::::::::::::::::

